
% begin of MD document

uvw\special{pdf:literal 1 0  0 1 0 -2 cm}xyz\special{pdf:literal 1 0  0 1 0 2 cm}abc\mktsShowpar\par
uvw\mktsUp{}12345678\mktsDown{}abc\mktsShowpar\par
\begin{multicols}{2}
\subsection{MKTS Markup: Three Layers
}
MingKwai TypeScript uses three layers of markup:\mktsShowpar\par
(1) A macro language which uses double pointy brackets—so called
  macro brackets—to escape into a meta-language;
(3) MarkDown as parsed by {\mktsStyleCode{}markdown-it}\footnote{https://github.com/markdown-it}
  (which basically means it should be quite close to the
  CommonMark\footnote{http://commonmark.org/}) specification); and
(3) HTML markup.\mktsShowpar\par
The ordering in the list above reflects the precedence of the layers:
the macro language takes the highest precedence and HTML constructs
the lowest. Because MarkDown proper takes higher precedence than
HTML, its constructs are parsed even if they appear between a pair
of HTML tags. Likewise, because the macro language takes the highest
priority, it is not possible to inhibit its interpretation by
putting a macro inside HTML tags or inside a MarkDown fenced code
block.\mktsShowpar\par
\begin{itemize}\item[$\star$] MKTS allows to use {\mktsStyleBold{}HTML} in the markup.\mktsShowpar\par

\item[$\star$] {\mktsStyleBold{}Regions} are used to markup stretches within a document,
be it inline spans, blocks of text or longer portions like
sections and chapters. Regions use parentheses to indicate
the start and end points:\mktsShowpar\par
\begin{itemize}\item[$\star$] Full Notation: {\mktsStyleCode{}<<(name>>...<<name)>>}
\item[$\star$] Short Notation: {\mktsStyleCode{}<<(name>>...<<)>>}
\end{itemize}When used in the long form, the name used in the start tag must
match the one in the end tag. As with HTML
tags, regions must be properly nested and must not overlap.\mktsShowpar\par

\end{itemize}\begin{itemize}\item[$\star$] {\mktsStyleBold{}Actions} allow to execute code snippets inside the document. They
come in two flavors: ‘silent’ and ‘vocal’.\mktsShowpar\par
Silent actions do not
leave a direct trace in the document unless their code calls an
API function to do that; vocal actions are replaced by their
evaluated value (i.e. whatever {\mktsStyleCode{}eval( "some code" )} returns).
\mktsShowpar\par
\begin{itemize}\item[$\star$] {\mktsStyleBold{}Silent Actions}
\item[$\star$] Full Notation: {\mktsStyleCode{}<<(.>>some code<<.)>>}, {\mktsStyleCode{}<<(.js>>some code<<.js)>>}
\item[$\star$] Short Notation: {\mktsStyleCode{}<<(.>>some code<<)>>}, {\mktsStyleCode{}<<(.js>>some code<<)>>}
\item[$\star$] {\mktsStyleBold{}Vocal Actions}
\item[$\star$] Full Notation: {\mktsStyleCode{}<<(:>>some code<<:)>>}, {\mktsStyleCode{}<<(:js>>some code<<:js)>>}
\item[$\star$] Short Notation: {\mktsStyleCode{}<<(:>>some code<<)>>}, {\mktsStyleCode{}<<(:js>>some code<<)>>}
\end{itemize}As with regions, the rule is that wherever the more explicit long form
is used, the action type marker ({\mktsStyleCode{}.} (dot) or {\mktsStyleCode{}:} (colon)) and the action
name of the start and end tags must match.\mktsShowpar\par

\item[$\star$] To execute an action without further logic code (i.e. a simple function call)  
in the manuscript, either a silent action {\mktsStyleCode{}<<(.>>makeitso 42<<.)>>} or the
short {\mktsStyleBold{}Command} notation {\mktsStyleCode{}<<!makeitso 42>>} can be used. Observe that
commands do not allow for a language annotation; whatever is inside
the macro tag will be parsed as CoffeeScript.\mktsShowpar\par

\item[$\star$] To interpolate the value of a variable into the document,
either a vocal action {\mktsStyleCode{}<<(:>>foo<<)>>} or the shorter {\mktsStyleBold{}Value}
notation {\mktsStyleCode{}<<\$foo>>} can be used.
\mktsShowpar\par

\item[$\star$] Finally, there are {\mktsStyleBold{}Raw} regions that give authors an opportunity
to talk directly to the \LaTeX typesetting system. Raw regions are
surrounded by triple pointy brackets, e.g. {\mktsStyleCode{}<<<...raw material...>>>}.\mktsShowpar\par

\end{itemize}\end{multicols}
% end of MD document
